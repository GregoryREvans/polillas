\documentclass[11pt]{article}
\usepackage{fontspec}
\usepackage[utf8]{inputenc}
\setmainfont{Bell MT Std}
\usepackage[paperwidth=11in,paperheight=8.5in,margin=1in,headheight=0.0in,footskip=0.5in,includehead,includefoot,portrait]{geometry}
\usepackage[absolute]{textpos}
\TPGrid[0.5in, 0.25in]{23}{24}
\parindent=0pt
\parskip=12pt
\usepackage{nopageno}
\usepackage{graphicx}
\graphicspath{ {./images/} }
\usepackage{amsmath}
\usepackage{tikz}
\newcommand*\circled[1]{\tikz[baseline=(char.base)]{
            \node[shape=circle,draw,inner sep=1pt] (char) {#1};}}

\begin{document}

\begin{textblock}{23}(0, 1)
\begin{center}
\huge FOREWORD
\end{center}
\end{textblock}

\vspace*{0.25\baselineskip}

\begingroup
\begin{center}
\leftskip0.5in
\textit{Moths} are the members of the Family \textit{Lepidoptera} which are not butterflies. Most moths are nocturnal, navigating through the skies with the aid of the moon and stars, sometimes falling astray of their desired path due to the intrusion of artificial light. The Bo\"otes void, sometimes called \textit{The Great Void} or \textit{The Great Nothing}, is a massive, practically barren region of outer space, known to contain 60 galaxies by 1997. Greg Aldering said of the void ``If the Milky Way had been in the centre of the Bo\"otes void, we wouldn't have know there were other galaxies until the 1960's.'' Taking this vacuum further, imagine a dark, black sky. Imagine a landscape of moths, torn from their signals of orientation, forced to swarm and beat against each other in search of home.
\rightskip\leftskip
\phantom{text} \hfill (GRE)

\vspace*{2\baselineskip}

\leftskip0.5in
Moth morning, maelstrom of mothwing at the window, fog-dulled sun trundled like a cart of moths across the aspirin sky. I stay in bed. Deer dance the bolero across the path. I used to count moths to sleep, each chewed-through cocoon an act not of transformation, but of violence against the past. The history of a moth is my history. Mesh-like, the world held me. I escaped. Or it did.
\rightskip\leftskip
\phantom{text}

[...]

\leftskip0.5in
One regret settles like a moth on my knuckle. I am in a room, whose wallpaper is thin rain. This is a dream or what is found in the mind's channel zero. The moth scans its head back and forth mechanically. It seems to be searching for the next place to land. Be quiet. The future is making a decision. It lifts, nestles in my ear. It says \textit{forgive me}.
\rightskip\leftskip
\phantom{text} \hfill (Andrew Grace -- Sancta [excerpts])
\end{center}
\endgroup

%\vspace*{2\baselineskip}

\begin{center}
\huge PERFORMANCE NOTES
\end{center}
\begingroup
\begin{center}

\leftskip0.25in
\pmb{String Contact Points} : The indications of string contact positions such as $sul \ tasto$ (abbreviated as $st.$), $sul \ ponticello$ (abbreviated as $sp.$), $molto \ sul \ ponticello$ (abbreviated as $msp.$), etc. should be considered as points along the continuum of the length string. The performer should make an effort to smoothly transition from one position to the next throughout the duration of the passage covered by the arrow-demarcated dashed line. When this arrow is not present, the performer should default to an $ordinario$ position. In extreme cases where this is a lack of space, sul tasto is abbreviated as \textit{T.} and sul ponticello is abbreviated as \textit{P.} 
\rightskip\leftskip
\phantom{text} \hfill \phantom{()}

\leftskip0.25in
\pmb{Bow Contact Points} : In various passages throughout this piece, there is notation which represents the point at which the bow is touched as it is drawn across the string. These positions are written as fractions where \( \frac{0}{7} \) and  \( \frac{0}{5} \) represent $au \ talon$ and \( \frac{7}{7} \) and \( \frac{5}{5} \) represent $punta \ d'arco$. For the duration of the note to which these fractions are attached, the performer should draw the bow at a constant speed, moving toward the destination point indicated on the following note. Bowings are provided. Passages without these indications should be bowed at the performer's discretion.
\rightskip\leftskip
\phantom{text} \hfill \phantom{()}

\leftskip0.25in
\pmb{Bow Rotation Indications} : \circled{1} $col \ legno \ tratto$ is abbreviated as $clt.$ and \circled{2} $col \ legno \ batutto$ is abbreviated as $clb.$. When these abbreviations are not present, the performer should default to ordinary $crine$ bowing techniques. \circled{3} $Twist$ is given as the direction to place the bow on two strings and to twist the bow in a circular motion to create a crackling sound. \circled{4} Circular bowing is indicated with an articulation symbol of a circle with an arrow. The speed of the circular motion is one full rotation for the duration under each articulation.
\rightskip\leftskip
\phantom{text} \hfill \phantom{()}

\pmb{Damping} : \circled{1} The damp sign (a miniature coda symbol) means three fingers placed harmonic-lightly on the string. The sound is `white' and damped, but with a pitch center to the band of white noise still very much discernible. \circled{2} The diamond symbols refer to two (or more depending on the number of diamonds) fingers placed harmonic-lightly on the string. \circled{3} The parenthetical-diamond spanners are trills between the single-, double- and triple-harmonic-damping techniques. That is: a double-diamond spanner with the top of the two diamonds parenthesized means to place two harmonic-light fingers on the string and then repeatedly lift-and-replace the finger closest to the nut on and off the string; this will effectively trill between vanilla-harmonic and the complex sound the double-harmonic will produce. Likewise, the triple-diamond spanner with the bottom two of the three diamonds parenthesized means to place three harmonic-light fingers on the string and then repeatedly lift-and-replace the bridgemost fingers on and off the string; this will effectively trill between the harmonic and the white-damped sound.
\rightskip\leftskip
\phantom{text} \hfill \phantom{()}

\pmb{Scordatura} : The lowest string of the Violoncello is to be tuned down one whole tone to B-flat. When this string is required, a bracket marked \textit{IV} is placed above the relevant passages. Passages performed on this string are transposed to the physical playing position on the string rather than the actual sounding pitch.
\rightskip\leftskip
\phantom{text} \hfill \phantom{()}

\leftskip0.25in
\pmb{Miscellaneous} : \circled{1} Tremoli should be performed as fast as possible and not as a measured subdivision of the duration to which they are attached. \circled{2} Diamond note heads represent a left hand finger pressure of a natural harmonic. Flageolet notation with a circle above the note head indicates a harmonic which may be achieved either as a natural or artificial harmonic. \circled{3} The choice to perform this piece either $senza \ vibrato$ or $con \ vibrato$ is left to the performers. \circled{4} $XFB$ represents an $extremely \ flautando \ bow$. $XFB$ tremoli should be slightly irregular, comprised of almost full bow strokes.
\rightskip\leftskip
\phantom{text} \hfill \phantom{()}

\leftskip0.25in
\pmb{Accidentals} : After temporary accidentals, cancellation marks are printed also in the following measure (for notes in the same octave) and, in the same measure, for notes in other octaves, but they are printed again if the same note appears later in the same measure, except if the note is immediately repeated. At times throughout the score, justly tuned intervals are indicated by the use of Helmholtz-Ellis notation combined with cent deviations from equal temperament for use with an electronic tuner. When no example pitch is given with the cent deviation, the mark is a deviation of the nearest ``standard'' accidental. If the performers wish to interpret the score without cent-tuning, the approximation of pitches to the nearest semi-tone is acceptable. When Helmholtz-Ellis notation is not given, the pitches are to be played as usual. The accidentals for Justly-intoned pitches are always present before the note head.
\rightskip\leftskip
\phantom{text} \hfill \phantom{()}
\end{center}
\endgroup

\vspace*{1\baselineskip}

\begin{center}
\textit{Polillas} was composed for the JACK Quartet as part of the 2021 residency at the University of Iowa.
\end{center}

\vspace*{1\baselineskip}

\begin{center}
duration: c. 23'
\end{center}

\end{document}
